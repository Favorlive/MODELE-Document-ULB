\chapter{Bibliographie}

\section{Livres}
Le choix des ouvrages et des références est porté sur des thématiques en lien avec les méthodes de gestion de projet. C'est naturellement vers la méthodologie Agile et Scrum que l'on porte la sélection. 

Les ouvrages privilégiés sont ceux qui comportent une partie pratique, notamment avec des exemples concrets, accompagné par de véritables témoignages. D'autres livres proposent de partir sur une équipe totalement fictive et d'appliquer des méthodes Agiles dans la gestion d'équipe et de projet.

L'illustration des théories par des exemples pratiques permettront de mieux les comparer avec notre cas d'étude avec quelques précaution toutefois. Les formules qui fonctionnent pour une entreprise en particulier ne fonctionneront pas forcément avec une autre. L'organisation, la culture d'entreprise et l'aspect humain sont des paramètres dont il est difficile d'apporter des manières de procéder fixes et issus de théories. Chaque organisation possède des équipes avec des dynamiques différents et chaque équipe est constituée de personnes avec chacun des tempéraments différents. (N.D.R., \textit{et fort heureusement d'ailleurs}).
\begin{itemize}
\item Cohn (Mike), Agile Estimating and Planning, Prentice Hall, 1 novembre 2005.
\item El Haddad (Bassem), Scrum, de la théorie à la pratique, Eyrolles, 7 novembre 2019. 
\item Messager (Véronique), Coacher une équipe agile. Guide à l’usage des ScrumMasters, managers et responsables de la transformation, Eyrolles, 7 septembre 2017. 
\item Messager (Véronique), Gestion de projet agile : avec Scrum, Lean, eXtreme Programming ..., Eyrolles, 2013
\item Morisseau (Laurent), Kanban : l’approche en flux pour l’entreprise agile, Dunod, 5 juin 2019.
\item Patton (Jeff), Le story mapping – visualiser vos user stories pour développer le bon produit, Dunod, 20 janvier 2020. 
\item Pradat-Peyre (Jean-François), Printz (Jacques), Pratique des tests logiciels. Concevoir et mettre en œuvre une stratégie de tests (3e édition), Collection Info-Pro-Etudes, 18 octobre 2017. 
\item Printz (Jacques), Architecture logicielles. Concevoir des applications simples, sûres et adaptables (3e édition), Collection InfoPro, 13 juin 2012. 
\item Subra (Jean-Paul), Scrum, une méthode agile pour vos projets, Collection DataPro, 5 septembre 2019.
\item Autissier (David), Moutot (Jean-Michel), Méthode de conduite du changement, Dunod, 2016, pp. 289-304. 
\end{itemize}


 
\section{Articles d’une revue scientifique}
Pour avoir une base plus rigoureuse, des articles scientifiques de revue viendront renforcer l'analyse du mémoire. Sont mis en avant les articles qui exposent les théories des méthodes de gestion de projet et qui apportent un regard critique sur les forces et les faiblesses. 
Ci-dessous un panel d'articles les plus pertinents parmi les résultats de la recherche. En d'autres termes, cette liste est amenée à évoluer et n'est pas exhaustive.
\begin{itemize}
\item Collignon (Alain), Schöpfel (Joachim), « Méthodologie de gestion agile d’un projet. Scrum – les principes de base », I2D – Information, données \& documents, volume 53, 2, 2016, pp. 12-15. 
\item Gentil (Pascale), Chédotel (Frédérique), « Outils et pratiques pour une compétence collective en situation, le cas de la méthode agile Scrum », Revue française de gestion, n°270, 2018, pp. 101 – 114. 
\item Séguin (Bruno Louis), Henrard (Jean-Christophe), Basset (Hervé), Collette (Fabrice), « Des outils pour toutes les plateformes », Documentaliste-Sciences de l'Information, volume 47, 2010, pp. 54 – 67. 
\item Deuff (Dominique), Cosquer (Mathilde), « Méthode agile centrée utilisateur », Proceedings of the 2012 Conference on Ergonomie et Interaction homme-machine, 2012, pp. 25-32. 
\end{itemize}

\section{Podcast, autres sources multimédia}
Le format podcast est souvent utilisé pour les interviews sur la thématique de gestion de Projet. Le contenu de ces interview est une source intéressante de cas pratiques. Ces sources ne seront pas utilisés à titre scientifique, mais pourront contribuer à nourrir les réflexions.

\begin{itemize}
\item L'Agilité racontée autrement (Women in agile, Paris);
\item Agile comme un panda
\item Parlons Design (Romain Penchenat)
\item Les voix du Design Thinking (Ines Beatrix)
\item Design+ (Laurent Gallen)
\item Scrum Life (Jean-Pierre Lambert)
\item La Minute Agile (Paquet Judicael)
\end{itemize}


\section{Autres sources, documents}
Le manifeste pour le développement Agile de logiciels, disponible sur \url{https://agilemanifesto.org/iso/fr/manifesto.html} 
